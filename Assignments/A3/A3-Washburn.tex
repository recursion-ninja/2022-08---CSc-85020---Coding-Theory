\documentclass{AssignmentCUNY}
\usepackage{amsmath}
\usepackage[]{amsthm} %lets us use \begin{proof}
\usepackage[]{amssymb} %gives us the character \varnothing
\usepackage[utf8]{inputenc}
\usepackage{enumitem}
\usepackage{mathrsfs}
\usepackage{nicematrix}
\usepackage{graphicx}
\usepackage{xifthen}
\usepackage{xspace}

\setlength{\abovedisplayskip}{1pt}
\setlength{\belowdisplayskip}{1pt}
\setlength{\abovedisplayshortskip}{1pt}
\setlength{\belowdisplayshortskip}{1pt}

\setcounter{MaxMatrixCols}{32}

\renewcommand{\thefootnote}{\fnsymbol{footnote}}

\newcommand{\CodeWord}[1]{\ensuremath{\langle\,#1\,\rangle}}
\newcommand{\CodeSchema}[4]{\ensuremath{\boldsymbol{\bigl(\bigr.}\,#1,~#2,~#3\,\boldsymbol{\bigl.\bigr)}_{#4}}}

\AssignmentNumber{3}%
\CourseName{Coding Theory}
\CourseNumber{85020}
\StudentName{Alex Washburn}

\begin{document}

\CoverPage

\Problem{%
Consider an LDPC code constructed with the bipartite graph below.
Show how the two flipped bits can be corrected.%
}%
\begin{figure}[h]
	\centering
%	\caption[Bipartite graph $\mathbf{G}$]{Bipartite graph $\mathbf{G}$ used for various codes described in problems 1 and 2 }
	\includegraphics[width=0.25\textwidth]{bipartite-graph.png}
\end{figure}

Suppose we transmit the message $\{\;m_1,\, m_2,\, m_3,\, m_4,\, m_5\;\} = 11110$ using the LDPC code constructed from $\mathbf{G}$.
The transmitted codeword is $\{\;m_1,\, m_2,\, m_3,\, m_4,\, m_5,\, p_1,\, p_2,\, p_3 \;\} = \CodeWord{11110000}$.
Let the $2^{nd}$ and $5^{th}$ bits of the codeword be flipped, and the received codeword being$ \{\;m_1,\, m_2,\, m_3,\, m_4,\, m_5,\, p_1,\, p_2,\, p_3 \;\} = \CodeWord{10111000}$. To decode and correct errors, we run the LDPC graph error correction algorithm on $\CodeWord{10111000}$:

\begin{enumerate}[label=\textbf{\emph{Round} \arabic*:}, leftmargin=*]

\item Are there unsatisfied parity bits?
\begin{align*}
	0 = p_1 \not =&\; m_1 \oplus m_3 \oplus m_5 = 1 \oplus 1 \oplus 1 = 1 \\
	0 = p_2      =&\; m_2 \oplus m_4 \oplus m_5 = 0 \oplus 1 \oplus 1 = 0 \\
	0 = p_3 \not =&\; m_3 \oplus m_4 \oplus m_5 = 1 \oplus 1 \oplus 1 = 1
\end{align*}
\[
\text{Yes, } \frac{2}{3} \text{ of parity bits unsatisfied!}
\]

\begin{enumerate}[label=\textbf{\emph{Step} \arabic*:}]

\item Find a message $m_n$ bit for which more than $\frac{d}{2}$ neighbors are unsatisfied:
\[
\text{Bit: } m_5 = 1\text{.}
\]

\item Flip that bit:
\[
\text{Bit: } m_5 = 0\text{.}
\]

\end{enumerate}

\item Are there unsatisfied parity bits?
\begin{align*}
	0 = p_1      =&\; m_1 \oplus m_3 \oplus m_5 = 1 \oplus 1 \oplus 0 = 0 \\
	0 = p_2 \not =&\; m_2 \oplus m_4 \oplus m_5 = 0 \oplus 1 \oplus 0 = 1 \\
	0 = p_3      =&\; m_3 \oplus m_4 \oplus m_5 = 1 \oplus 1 \oplus 0 = 0
\end{align*}

\[
\text{Yes, } \frac{1}{3} \text{ of parity bits unsatisfied!}
\]

\begin{enumerate}[label=\textbf{\emph{Step} \arabic*:}]

\item Find a message $m_n$ bit for which more than $\frac{d}{2}$ neighbors are unsatisfied:
\[
\text{Bit: } m_2 = 0\text{.}
\]

\item Flip that bit:
\[
\text{Bit: } m_2 = 1\text{.}
\]	

\end{enumerate}

\item Are there unsatisfied parity bits?
\begin{align*}
	0 = p_1      =&\; m_1 \oplus m_3 \oplus m_5 = 1 \oplus 1 \oplus 0 = 0 \\
	0 = p_2      =&\; m_2 \oplus m_4 \oplus m_5 = 1 \oplus 1 \oplus 0 = 0 \\
	0 = p_3      =&\; m_3 \oplus m_4 \oplus m_5 = 1 \oplus 1 \oplus 0 = 0
\end{align*}
\[
\text{No, halt and output the message } 11110\text{.}
\]

\end{enumerate}


\Problem{%
Consider a Tornado code constructed with the same bipartite graph.
Show how the message with two erased symbols can be decoded.%
}%

Suppose we transmit the message $\{\;m_1,\, m_2,\, m_3,\, m_4,\, m_5\;\} = 11111$ using a Tornado code constructed from $\mathbf{G}$.
The transmitted codeword is $\{\;m_1,\, m_2,\, m_3,\, m_4,\, m_5,\, p_1,\, p_2,\, p_3 \;\} = \CodeWord{11111111}$.
Let the $2^{nd}$ and $5^{th}$ bits of the codeword be erased, and the received codeword being$ \{\;m_1,\, m_2,\, m_3,\, m_4,\, m_5,\, p_1,\, p_2,\, p_3 \;\} = \CodeWord{1?11?111}$. To decode and recover erasures, we run the Tornado code decoding algorithm on $\CodeWord{1?11?111}$:

\begin{enumerate}[label=\textbf{\emph{Round} \arabic*:}, leftmargin=*]

\item Are there any erased bits?
\[\text{Bits: } \CodeWord{1?11?111} \]
\[\text{Yes, } m_2 \text{ and } m_5 \text{.}\]

\begin{enumerate}[label=\textbf{\emph{Step} \arabic*:}]
	
\item Find a parity bit such that only one of its neighbors is erased:
\[p_1 \to m_5\]

\item Fix the erased bit:
\begin{align*}
p_1 &= m_1 \oplus m_3 \oplus m_5 \\
  1 &=     1 \oplus 1 \oplus m_5 \\
  0 &=              1 \oplus m_5 \\
  1 &= m_5
\end{align*}

\end{enumerate}

\item Are there any erased bits?
\[\text{Bits: } \CodeWord{1?111111} \]
\[\text{Yes, } m_2 \text{.}\]

\begin{enumerate}[label=\textbf{\emph{Step} \arabic*:}]

\item Find a parity bit such that only one of its neighbors is erased:
\[p_2 \to m_2\]

\item Fix the erased bit:
\begin{align*}
p_1 &= m_2 \oplus m_4 \oplus m_5 \\
  1 &= m_2 \oplus 1 \oplus 1 \\
  0 &= m_2 \oplus 1 \\
  1 &= m_2
\end{align*}

\end{enumerate}

\item Are there any erased bits?
\[\text{Bits: } \CodeWord{11111111} \]
\[\text{No, halt and output the message } 11111\text{.}\]

\end{enumerate}


\Problem{%
Consider an $\CodeSchema{n}{k}{n-k+1}{q}$ RS code where $q = 2^{b}$ . How many bits can there be
at most in a decodable burst error?
}%

\newcommand{\Burst}[2]{%
\ensuremath{\textsc{Burst}\left(\,#1,\,#2\,\right)}\xspace%
}%

Note the form of the RS code construction $\CodeSchema{n}{k}{d}{q} = \CodeSchema{n}{k}{n - k + 1}{2^b}$ implies $d = n - k + 1$.
Recall that the number of correctable errors is equal to $\lfloor\frac{d - 1}{2}\rfloor$.
Therefore we have the number of possible corrections equal to:
\[
\left\lfloor\frac{d - 1}{2}\right\rfloor = \left\lfloor\frac{n - k + 1 - 1}{2}\right\rfloor = \left\lfloor\frac{n - k}{2}\right\rfloor \;\text{ alphabet symbols} \in \Sigma
\]

The alphabet of symbols $\Sigma$ has cardinality $q = 2^{b}$.
We can easily see this means there exists an isomorphism $\mathcal{I}$ mapping each symbol $s \in \Sigma$ to a $b$-bit binary encoding. 

Let us define \Burst{e}{i} to be a ``burst error'' of $e$ consecutive bits starting at bit $i$, where the bit flips occur at bits $i + 0,\, i + 1,\, \hdots\, ,\, i + e - 1$.

Suppose a message $m$ is sent with $k$ symbols from $\Sigma$ after being converted to a codeword $c$ with $n$ symbols from $\Sigma$:
\[
c = \{\;c_0,\, c_1,\, \hdots\,,\, c_{n-1} \;\}
\]

The codeword $c \in \Sigma^n$ can be transformed via $\mathcal{I}$ to it's binary encoding $e \in \{0,1\}^{n*b}$.

\begin{align*}
\mathcal{I}(c) &= \{\; \mathcal{I}(c_0),\, \mathcal{I}(c_1),\, \hdots\hdots\,,\, \mathcal{I}(c_{n-1}) \;\} \\
&= \{\; e_0,\, e_1,\, \hdots\,,\, e_{b-1},\, \mathcal{I}(c_1),\, \hdots\hdots\,,\, \mathcal{I}(c_{n-1}) \;\} \\
&= \{\; e_0,\, e_1,\, \hdots\,,\, e_{b-1},\,e_{b},\, e_{b+1},\, \hdots\,,\, e_{2b},\,
 \hdots\hdots\,,\, \mathcal{I}(c_{n-1}) \;\} \\
&= \{\;
e_0,\, e_1,\, \hdots\,,\, e_{b-1},\,
e_{b},\, e_{b+1},\, \hdots\,,\, e_{2b},\,
\hdots\hdots\,,\, e_{(n-2)b},\, e_{(n-2)b+1},\, \hdots\,,\,e_{(n-1)*b} \;\}
\end{align*}


If we have \Burst{b}{i}, where $i \mod b \equiv 0$, then the $b$ errors of \Burst{b}{i} reside entirely with \emph{one} symbol in $\Sigma$ and require only \emph{one} error correction from the RS code to remedy all $b$ errors.
However, by the pigeonhole principle, the $b+1$ errors of \Burst{b+1}{i} cannot reside in a single symbol in $\Sigma$ and requires at least two error corrections from the RC code to recover the $b+1$ errors.\\

Therefore the RS code can handle at most $b * \left\lfloor\frac{n - k}{2}\right\rfloor$ error bits in a burst error.


%Notes on block codes:
%
%\begin{align}
%\text{Notation:      }& \left( n, k, d \right)_{q}\\
%\text{Alphabet:      }& \Sigma\\
%\text{Codewords:     }& \mathbf{C} \subset \Sigma^{n}\\
%\text{Messages:      }& \mathbf{M} \subset \Sigma^{k}\\
%\text{Codeword Size: }& n \\
%\text{Message  Size: }& k \\
%\text{Distance:      }& d = \min\left\{ \Delta(x,y) : x,y \in \mathbf{C}\;\; x \not= y \right\}\\
%\text{Symbols:       }& q = \left\lvert\Sigma\right\rvert
%\end{align}



\end{document}
